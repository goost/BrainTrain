For english readers\+:~\newline
 This O\+S\+G-\/\+Demo was done as part of the computer graphics I 2014 lecture of the University of Applied Sciences and Arts Hanover. This page is a short description of the project which was a requirement for passing the lecture.~\newline
 Everything written here can either also be read in the source documentation or experienced by oneself by trying out this project (look for the hidden room!)~\newline
 Everything here uses the G\+N\+U Public License. Use as you want, but give us credit =)!~\newline
 ~\newline
 Das Projekt \char`\"{}\+Brain\+Train\char`\"{} wurde im Rahmen der Vorlesung C\+G1 an der H\+S Hannover im Sommersemester 2014 erstellt.~\newline
 Beteiligt waren hierbei\+: Jonathan Spielvogel, Marcel Felix, Gleb Ostrowski, Phillip Sauer und Sebastian Huettermann.\hypertarget{index_Die}{}\section{Szene}\label{index_Die}
Ziel war es eine alte, unfertige, verfallene U-\/\+Bahn-\/\+Station zu entwerfen, in der der \char`\"{}\+Spieler\char`\"{} sich im First-\/\+Person-\/\+Shooter-\/\+Stil frei bewegen kann. Der Spieler startet am Kopf eines kleinen Niederganges, bestehend aus Treppe mit Gelaender und Rolltreppe. Direkt hinter der Startposition des Spielers befindet sich ein \char`\"{}geheimer Raum\char`\"{} (hier kann durch die Wand gelaufen werden). Der untere Teil besteht aus einem einzelnen, geschwungenen Bahnsteig (mit einem Gleis). Auf dem Bahnsteig selber befinden sich diverse Modelle, z.\+B\+: Kisten, ein altes Tickethaeuschen mit einer Folie darueber, Oelfaesser mit einer sowjetischen Flagge darueber, einem Fliesenspiegel und viele mehr. Es gibt also viel zu entdecken!

Nicht in Blender wurde hierbei folgendes erstellt\+:
\begin{DoxyItemize}
\item Die Sitzbaenke wurden in O\+S\+G modelliert, materialisiert und texturiert
\item Die Bierflaschen sowie die Vase und Blume (auf dem Tickethaeuschen) und Figuren im geheimen Raum wurden als Rotationskoerper realisiert.
\begin{DoxyItemize}
\item Die Bierflaschen wurdem zudem mit einem Partikelsystem aus Blender (zufaellig) im Raum verteilt. Die primaere Farbgebung (der \char`\"{}\+Cartoon Effekt\char`\"{} inklusive Cel Shading und Nebel) wurde hierbei ueber eigene Shader implementiert. Zudem verfuegt der Spieler ueber ein \char`\"{}\+Waffen\+H\+U\+D\char`\"{}, das eine der weiteren Kamera darstellt.
\end{DoxyItemize}
\end{DoxyItemize}\hypertarget{index_Animationen}{}\section{Animationen}\label{index_Animationen}
Mehrere Dinge sind animiert\+: Zum einen faehrt in regelmaessigen Abstaenden ein Zug das Gleis entlang. Diese Animation wurde als Animation Path in O\+S\+G realisiert. Und zum anderen weht die grosse, haengende Flagge \char`\"{}im Wind\char`\"{}. Diese Animation wurde ueber den Shader realisiert. Ebenso ueber die Shader sind die Animationen der Figuren im geheimen Raum realisiert.\hypertarget{index_Interaktive}{}\section{Elemente}\label{index_Interaktive}
In der gesamten Szene kann mit einigen Elementen interagiert werden\+:
\begin{DoxyItemize}
\item Am rechten Ende des Bahnhofs kann von einer Kiste eine \char`\"{}kaputte Portalgun\char`\"{} mittels Links-\/\+Klick aufgehoben werden. Sobald der Spieler mehr als eine Waffe traegt kann diese mittels Scrollen des Mausrades gewechselt werden.
\item Einige (drei) der herumliegenden Bierflaschen koennen getrunken werden. Hierzu muss der Spieler sich im geduckten Modus der Flasche naehern und kann diese -\/ sofern der entsprechende Texthinweis erscheint -\/ mit einem Links-\/\+Klick trinken. Hierbei handelt es sich um einen sehr starken Alkohol, der zwar schnell wirkt, seine Wirkung aber auch schnell wieder verliert.
\begin{DoxyItemize}
\item Aus Sicherheitsgruenden darf leider nicht verraten werden, um welche Flaschen es sich handelt.
\end{DoxyItemize}
\item Im Geheimraum am oberen Ende der Treppe stehen einige farbige Figuren. Naehert sich der Spieler diesen, so kann er mit einem Links-\/\+Klick u.\+a. Shader-\/\+Farben wechseln. Probieren Sie es aus!
\end{DoxyItemize}\hypertarget{index_Die}{}\section{Szene}\label{index_Die}
Die Bewegung in der Szene erfolgt im gewohnten F\+P\+S Stil. Hierbei ist sowohl eine Kollisionserkennung als auch eine \char`\"{}\+Clamp to Ground\char`\"{} Funktionalitaet implementiert (so dass der Spieler auf dem Boden laeuft). Die Tastaturbelegung ist hierbei die folgende\+:
\begin{DoxyItemize}
\item {\bfseries W} Bewegung Vorwaerts, {\bfseries S} Bewegung Rueckwaerts, {\bfseries A} Nach links bewegen (nicht drehen), {\bfseries D} Nach rechts bewegen (nicht drehen)
\item {\bfseries Maus\+:} Umschauen
\item {\bfseries Mausklick} links zur Interaktion (es erscheint immer ein Text der Interaktion \char`\"{}ankuendigt\char`\"{})
\item {\bfseries Leerstaste\+:} Springen, {\bfseries L-\/\+Shift}\+: Sprinten, {\bfseries L-\/\+Strg}\+: Gehen (langsamer gehen), {\bfseries X\+:} Ducken (um z.\+B. an Bierflaschen heranzukommen)
\item {\bfseries Mausrad} {\bfseries scrollen\+:} Wechseln der Waffe (sofern mehr als eine getragen wird) Mit der Taste {\bfseries F} kann in den Flugmodus gewechselt werden. In diesem ist die Kollisionserkennung nicht mehr aktiv. Zu der o.\+g. Steuerung kommt Folgendes hinzu\+:
\item {\bfseries Q} senkrecht nach unten fliegen, {\bfseries E} senkrecht nach oben fliegen
\end{DoxyItemize}

Weiteres\+:
\begin{DoxyItemize}
\item {\bfseries C} aktiviert/deaktiviert den Polygon-\/\+Modus
\item {\bfseries L-\/\+Shift} {\bfseries +} {\bfseries 1} wechselt durch die Shader-\/\+Modi, die normalerweise im Geheimraum umgeschaltet werden koennen
\end{DoxyItemize}\hypertarget{index_Quellenverzeichnis}{}\section{Quellenverzeichnis}\label{index_Quellenverzeichnis}
Sofern nicht anders dokumentiert (z.\+B. im Quellcode), handelt es sich bei allen Entwicklungen um Eigenentwicklungen. Insbesondere sind saemtliche Modelle Eigenentwicklungen.~\newline
 Texturen kommen hierbei geschlossen von~\newline
 \href{http://www.cgtextures.com/}{\tt http\+://www.\+cgtextures.\+com/} ~\newline
 Ausnahmen sind hierbei\+:~\newline
 U\+D\+S\+S\+R Flagge (auf den alten Oelfaessern liegend)~\newline
 \href{http://freestock.ca/soviet_union_ussr_grunge_flag_sjpg1191.jpg}{\tt http\+://freestock.\+ca/soviet\+\_\+union\+\_\+ussr\+\_\+grunge\+\_\+flag\+\_\+sjpg1191.\+jpg} ~\newline
 Zuletzt geprueft/gesehen\+: 19.\+06.\+2014~\newline
 Flagge mit Einhorn (in der Ecke haengend)~\newline
 \href{http://wallpoper.com/images/00/24/35/71/communism-soviet_00243571.jpg}{\tt http\+://wallpoper.\+com/images/00/24/35/71/communism-\/soviet\+\_\+00243571.\+jpg} ~\newline
 Zuletzt geprueft/gesehen\+: 19.\+06.\+2014~\newline
